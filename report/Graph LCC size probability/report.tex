\documentclass{article}

\usepackage[english]{babel}
\usepackage[utf8]{inputenc}
\usepackage[T1]{fontenc}
\usepackage{fullpage}
\usepackage[parfill]{parskip}
\usepackage{tikz}
\usepackage{float}

\usepackage{palatino, eulervm}

% math
\usepackage{amsmath}
\usepackage{amssymb}
\usepackage{amsfonts}
\usepackage{amsthm}
\usepackage{commath}
\usepackage{stmaryrd}
\usepackage{mathtools}
\usepackage{mathrsfs}
\usepackage[framemethod=tikz]{mdframed}

\newtheorem{lemma}{Lemma}[section]
\newtheorem{corollary}[lemma]{Corollary}
\newtheorem{proposition}[lemma]{Proposition}
\newtheorem{theorem}[lemma]{Theorem}
\newtheorem{conjecture}[lemma]{Conjecture}
\theoremstyle{definition}
\newtheorem{definition}[lemma]{Definition}
\theoremstyle{remark}
\newtheorem*{remark}{Remark}
 % operators
\DeclareMathOperator{\LCC}{LCC}
\DeclareMathOperator*{\argmax}{arg\,max}
\DeclareMathOperator*{\argmin}{arg\,min}

 % commands
\renewcommand{\P}{\mathbb P}

\newcommand{\R}{\mathbb R}
\newcommand{\N}{\mathbb N}
\newcommand{\intint}[2]{\left\llbracket#1, \, #2\right\rrbracket}
\newcommand{\floor}[1]{\left\lfloor#1\right\rfloor}
\newcommand{\ceil}[1]{\left\lceil#1\right\rceil}
\newcommand{\st}{\text{ s.t. }}

% link amsthm and mdframed
\iftrue
%\iffalse
	% pre-amsthm
	\mdfdefinestyle{resultstyle}{%
		hidealllines=true,%
		leftline=true,%
		rightline=true,%
		innerleftmargin=10pt,%
		innerrightmargin=10pt,%
		innertopmargin=10pt,%
		innerbottommargin=8pt,%
	}

	\surroundwithmdframed[style=resultstyle]{theorem}
	\surroundwithmdframed[style=resultstyle]{proposition}
	\surroundwithmdframed[style=resultstyle]{corollary}
	\surroundwithmdframed[style=resultstyle]{lemma}
\fi

\author{Robin Petit}
\date{March-April 2017}
\title{On the distribution of the largest connected component size in random graphs with fixed edges set size}

\begin{document}
\maketitle
\tableofcontents

\section{Introduction}
	\subsection{Definitions and preliminary results}
		Let's consider $V = \{v_1, \ldots, v_{\abs V}\}$ a set of vertices. We denote by $\abs V$ the cardinality of the set $V$. Let's define
		the function:
		\[X : \N \to \N : n \mapsto \frac {n(n-1)}2.\]

		\begin{definition} An undirected graph $\Gamma$ is denoted $\Gamma(V, E)$ for $V$ its vertices set, and $E$ its edges set, with $E = \{e_1, \ldots, e_{\abs E}\}$
		and $\forall i \in \intint 1{\abs E} : e_i = \{v_{i1}, v_{i2}\}$ for $1 \leq i_1, i_2 \leq \abs V$ with $i_1 \neq i_2$ (i.e. loops are not tolerated).
		\end{definition}

		\begin{remark} $\abs E$ is usually denoted as $m$, and $\abs V$ is sometimes denoted as $n$. Both these numbers are (non-strictly) positive integers.
		\end{remark}

		\begin{definition} The set of all the existing graphs having given vertices set $V$ is denoted by $\Gamma(V, \cdot)$. We denote $\Gamma_m(V, \cdot)$
		the subset of $\Gamma(V, \cdot)$ such that $\abs E = m$.
		\end{definition}

		\begin{remark} We observe that:
		\[\Gamma(V, \cdot) = \bigsqcup_{m \in \N}\Gamma_m(V, \cdot).\]
		\end{remark}

		\begin{definition} For every $n \in \N$, we define $\mathcal K_n$ as the \textit{complete graph} of size $n$.
		\end{definition}

		\begin{lemma} For a graph $\Gamma(V, E)$, we have $\abs E \leq X(\abs V)$.
		\end{lemma}

		\begin{proof} We know that $\Gamma(V, E) \leq \mathcal K_{\abs V}$, and $\mathcal K_{\abs V}$ has exactly $X(\abs V)$ edges (vertex $v_i$ is connected
		to vertices $v_{i+1}$ to $v_{\abs V}$, so the number of edges is equal to $\sum_{i=1}^{\abs V}(\abs V - i) = \sum_{i=0}^{\abs V - 1}i = X(\abs V)$).
		\end{proof}

		\begin{lemma} For given vertices set $V$ and fixed number of edges $m \in \N$, we have:
		\[\abs {\Gamma_m(V, \cdot)} = \begin{cases}\binom {X(\abs V)}m &\text{ if } m \leq X(\abs V) \\0 &\text{ else}\end{cases}.\]
		\end{lemma}

		\begin{corollary} For given vertices set $V$, we have $\abs {\Gamma(V, \cdot)} = 2^{X(\abs V)}$.
		\end{corollary}

		\begin{proof} Since $\Gamma(V, \cdot)$ is given by a disjoint union over $m$, its cardinality is equal to the sum of the individual cardinalities:
		\[\abs {\Gamma(V, \cdot)} = \sum_{m \in \N}\abs {\Gamma_m(V, \cdot)} = \sum_{k=0}^{X(\abs V)}\abs {\Gamma_m(V, \cdot)} = \sum_{k=0}^{X(\abs V)}\binom {X(\abs V)}m = 2^{X(\abs V)}.\]
		\end{proof}

		\begin{definition} A graph $\Gamma(V, E)$ is said to be connected if for each $v, w \in V$, there exists a path between $v$ and $w$. We denote by $\chi(V, \cdot)$ the
		set of all connected graphs having vertices set $V$. Again, for $m \in \N$, we denote by $\chi_m(V, \cdot) \subset \chi(V, \cdot)$ the set of connected graphs having $m$ edges.
		\end{definition}

		\begin{remark} $\chi(V, \cdot) \subset \Gamma(V, \cdot)$, and:
		\[\chi(V, \cdot) = \bigsqcup_{m \in \N}\chi_m(V, \cdot).\]
		\end{remark}

		\begin{lemma} For $m < \abs V$ or $m > X(\abs V)$, we have $\abs {\chi_m(V, \cdot)} = 0$.
		\end{lemma}

		\begin{definition}\label{def:mu function} Let's define the function:
		\[\mu : \mathcal P(V) \to \intint 1{\abs V} : W \mapsto \mu(W) \coloneqq \inf\; \{i \in \intint 1{\abs V} \st v_i \in W\}.\]
		\end{definition}

		\begin{definition} For every $W \in \mathcal P(V)$, we define $\Delta_W : \Gamma(V, \cdot) \to \Gamma(W, \cdot) : \Gamma(V, E) \mapsto \Gamma'(W, E')$
		such that:
		\[E' = \left\{\{v_i, v_j\} \in E \st v_i, v_j \in W\right\}.\]
		\end{definition}

		\begin{definition} We define the \textit{connected component of vertex $v_i \in V$ in graph $\Gamma(V, E)$} by the biggest subset (in the sense of inclusion) $W$
		of $V$ such that $v_i \in W$ and $\Delta_W(\Gamma(V, E)) \in \chi(W, \cdot)$.

		For graph $\Gamma(V, E) \in \Gamma(V, \cdot)$, we define $\abs {\LCC(\Gamma(V, E))}$ by:
		\[\abs {\LCC(\Gamma(V, E)} \coloneqq \max_{W \in \mathcal P(V)}\abs W\mathbb I_{\left[\Delta_W(\Gamma(V, E) \in \chi(V, \cdot)\right]}.\]

		We then define the \textit{largest connected component of the graph $\Gamma(V, E)$} as:
		\[\LCC(\Gamma(V, E)) \coloneqq \argmin_{\stackrel {W \in \mathcal P(V)}{\abs W = \abs {\LCC(\Gamma(V, E))}}} \mu(W).\]

		The set $\Lambda_k^m(V, \cdot)$ is then the set of all graphs $\Gamma(V, E) \in \Gamma(V, \cdot)$, such that $\abs E = m$ and $\abs {\LCC(\Gamma(V, E))} = k$.
		\end{definition}

		\begin{remark} The notations here are consistent since $\abs {\LCC(\Gamma(V, E))}$ corresponds indeed to the cardinality of $\LCC(\Gamma(V, E))$.

		Furthermore, this definition of largest connected component allows to define uniquely the LCC, even though a graph $\Gamma(V, E)$ has several connected component
		of same size. For example, following graph has two connected component of size 2, i.e. $\{1, 2\}$ (in red) and $\{3, 4\}$ (in blue).

		% Plot graph with 4 nodes and two non-adjacent edges
		\begin{figure}[H]
		\begin{center}
			\begin{tikzpicture}
				\node[shape=circle, draw=black](1) at (0, 3) {1};
				\node[shape=circle, draw=black](2) at (3, 3) {2};
				\node[shape=circle, draw=black](3) at (3, 0) {3};
				\node[shape=circle, draw=black](4) at (0, 0) {4};

				\path[draw=red] (1) -- (2);
				\path[draw=blue] (3) -- (4);
			\end{tikzpicture}
		\end{center}
		\caption{Graph $\Gamma\left(\{1, 2, 3, 4\}, \left\{\{1, 2\}, \{3, 4\}\right\}\right)$}
		\end{figure}

		Nevertheless, the $\LCC$ operator yields $\{1, 2\}$ since it minimizes the lowest id of element in connected component ($1$ for this graph).
		\end{remark}

		\begin{remark} Since $\Lambda_k(V, \cdot) = \bigsqcup_{m=0}^{X(\abs V)}\Lambda_k^m(V, \cdot)$ and:
		\[\Gamma(V, \cdot) = \bigsqcup_{k=1}^{\abs V}\bigsqcup_{m=0}^{X(\abs V)}\Lambda_k^m(V, \cdot),\]
		we want to know what is $\abs {\Lambda_k^m(V, \cdot)}$ equal to.
		\end{remark}

		\begin{definition} Let's declare a new random variable $\mathscr G(V, m)$, a graph uniformly distributed in $\Gamma_m(V, \cdot)$, thus such that:
		\[\forall \Gamma(V, E) \in \Gamma_m(V, \cdot) : \P[\mathscr G(V, m) = \Gamma(V, E)] = \frac 1{\abs {\Gamma_m(V, \cdot)}} = \frac 1{\binom {X(\abs V)}m}.\]
		\end{definition}

	\subsection{Objectives}
		The objective now is to find an expression for $\abs {\Lambda_k(V, \cdot)}$ since we are looking for:
		\[\P\left[\abs {\LCC(\mathscr G(V, m))} = k\right] = \frac {\abs {\Lambda_k(V, \cdot)}}{\abs {\Gamma(V, \cdot)}}
			= \frac 1{\abs {\Gamma(V, E)}}\sum_{m=0}^{X(\abs V)}\abs {\Lambda_k^m(V, \cdot)}.\]

		Let's denote this value $p_k \coloneqq \P\left[\abs {\LCC(\mathscr G(V, m))} = k\right].$

\section{Results}\label{sec:results}
	The general idea in order to determine $\abs {\Lambda_k^m(V, \cdot)}$ is to insert a connected component of size $k$ on vertices set $V$, and then to tally the configurations
	placing $m-k$ vertices without making a bigger connected component than the first one.

	\subsection{Examples}
		\subsubsection{$\abs {\Lambda_{k=1}(V, \cdot)}$}
			It is trivial to tell $\abs {\Lambda_1^m(V, \cdot)} = \delta_0^m$, i.e. equals one if $m=0$ and equals zero if $m>0$: a graph having at least one edge, cannot have
			a largest connected component of size 1.

		\subsubsection{Upper boundary of $m$ for $\abs {\Lambda_k^m(V, \cdot)}$}
			\begin{lemma}[Upper boundary of edges amount for $k=2$]\label{lemma:upper boundary k=2} For $m > \frac {\abs V}2$, we have $\Lambda_2^m(V, \cdot) = \emptyset$.
			\end{lemma}

			\begin{proof} To have a largest connected component of size $2$, each vertex must have degree $0$ or $1$. Take $m \in \N$ such that $m > \frac V2$. Take
			$\Gamma(V, E)$ such that $\abs E = m$, and take $\mathcal V_1 \coloneqq \{v \in V \st \deg(v) \leq 1\} \subset V$. Take the restriction
			$\Gamma'(\mathcal V_1, E') = \Delta_{\mathcal V_1}(\Gamma(V, E))$.

			Since in a graph, the sum of the degree of each vertex is equal to twice the amount of edges, when applied on $\Gamma'$, it follows that:
			\[2\abs {E'} = \sum_{v \in \mathcal V_1}\deg(v) \leq \sum_{v \in \mathcal V_1}1 = \abs {\mathcal V_1}.\]

			We then deduce that $\abs {E'} \leq \frac {\abs {\mathcal V_1}}2 \leq \frac {\abs V}2$. Thus $\mathcal V_1$ must be \textit{strictly} included in $V$,
			and then there must exist $v \in V$ such that $\deg(v) \geq 2$. Thus:
			\[\forall m > \frac {\abs V}2 : \forall \Gamma(V, E) \in \Gamma_m(V, \cdot) : \Gamma(V, E) \not \in \Lambda_2^m(V, \cdot).\]
			\end{proof}

			\begin{lemma} For $\Gamma(V, E) \in \Gamma(V, \cdot)$ a graph and $k \in \intint 1{\abs V}$, if there exists a vertex $v \in V$ such that $\deg(v) = k$,
			then $\abs {\LCC(\Gamma(V, E))} \geq k+1$.
			\end{lemma}

			\begin{proof} Take $v \in V$ such that $\deg(v) = k$. There exist $\{v_{i_1}, \ldots, v_{i_k}\} \subset V$ such that:
			\[\forall j \in \intint 1k : \{v, v_{i_j}\} \in E.\]

			Thus $\{v, v_{i_1}, \ldots, v_{i_k}\}$ is a connected component of size $k+1$. Thus the largest connected component must have size at least that big.
			\end{proof}

			\begin{proposition}[Upper boundary of edges amount generalized] For $k \in \intint 1{\abs V}$, and $m > \frac {\abs V(k-1)}2$, we have $\Lambda_k^m(V, \cdot) = \emptyset$.
			\end{proposition}

			\begin{proof} Take $m > \frac {(k-1)\abs V}2$, and $\Gamma(V, E) \in \Gamma_m(V, \cdot)$. Take $\mathcal V_k \coloneqq \{v \in V \st \deg(v) \leq k-1\}$.
			Let $\Gamma'(\mathcal V_k, E')$ be defined by $\Delta_{\mathcal V_k}(\Gamma(V, E))$. We know that:
			\[2\abs {E'} = \sum_{v \in \mathcal V_k}\deg(v) \leq (k-1)\abs {\mathcal V_k} \leq (k-1)\abs V.\]

			We deduce that $\abs {E'} \leq \frac {(k-1)\abs V}2 < m = \abs E$. Thus $\abs E \gneqq \abs {E'}$, and this implies that there exists $v \in V$ such that $\deg(v) \geq k$.
			By previous lemma, largest connected component size must be at least $k+1$.
			\end{proof}

			\begin{remark} We can understand this upper boundary as $m > \frac {\abs V(k-1)}2 = \frac {\abs V}k\frac {k(k-1)}2 = \frac {\abs V}k \cdot X(k)$. So in
			order to have a LCC of size $k$, edges can be distributed to make $\floor {\frac {\abs V}k}$ complete graphs having each $X(k)$ edges.
			The maximum amount of edges is then given by $\frac {\abs V(k-1)}2$.
			\end{remark}

		\subsubsection{$\abs {\Lambda_{k=2}(V, \cdot)}$}
			Example of size 2 is a bit more complicated:
			\[\forall m \in \intint 1{\floor {\frac {\abs V}2}} : \abs {\Lambda_2^m(V, \cdot)} = \begin{cases}
				\frac 1{m!}\prod_{k=0}^{m-1}\binom {\abs V-2k}2 &\text{ if } m \leq \frac {\abs V}2 \\
				0 &\text{ else}
			\end{cases}.\]

			\begin{proof} For $m > \frac {\abs V}2$, result is shown in Lemma~\ref{lemma:upper boundary k=2}. The part $\prod_{k=0}^{m-1}\binom {\abs V-2k}2$ corresponds
			to the choice of $m$ edges without making a connected component of size $\geq 3$.

			$\binom {\abs V-2 \cdot 0}2$ is the choice of the first edge (two vertices) among $\abs V$ vertices, $\binom {\abs V-2}2$ is the choice of the second edge
			(two vertices) among the $\abs V-2$ vertices left, etc. At step $\ell$, only $\abs V-2(\ell-1)$ vertices are available because two are selected per step, and
			a selected vertex cannot be used again, otherwise its degree would be $\geq 2$, and then the largest component size would be $\geq 3$.

			The $\frac 1{m!}$ comes from the fact that the order the edges are selected doesn't matter (so for each choice of $m$ edges, there are $m!$ permutations
			of these).
			\end{proof}

			\begin{remark} This can also be expressed as:
			\[\abs {\Lambda_2^m(V, \cdot)} = \frac 1{m!}\frac {\abs V!}{2^m\left(\abs V-2m\right)!},\]
			by simplification of the product.
			\end{remark}

\section{Processing on examples}
	\begin{align*}
		\abs {\Lambda_3^0(V, \cdot)} = \abs {\Lambda_3^1(V, \cdot)} &= 0 \\
		\abs {\Lambda_3^2(V, \cdot)} &= \binom {\abs V}3\binom 32 \\
		\abs {\Lambda_3^3(V, \cdot)} &= \binom {\abs V}3\binom 33 + \binom {\abs V}3\binom 32\binom {\abs V-3}2 \\
		\abs {\Lambda_3^4(V, \cdot)} &= \binom {\abs V}3\binom 33\binom {\abs V-3}2 + \binom {\abs V}3\binom 32\binom {\abs V-3}3\binom 32 + \binom {\abs V}3\binom 32\binom {\abs V-3}2\binom {\abs V-5}2.
	\end{align*}

	\begin{definition} Let's denote equally $\abs {\Lambda_k^m(n)} = \Lambda_k^m(n) \equiv \abs {\Lambda_k^m(V, \cdot)}$ for $V$ such that $\abs V = n$.
	\end{definition}

	This notation allows to lighten the expressions.

	\begin{remark} Current explorations tend to a formula looking something like:
	\[\abs {\Lambda_k^m(\abs V)}
		= \binom {\abs V}k\sum_{\ell=k-1}^{\min\left(m, X(k)\right)}\abs {\Lambda_k^\ell(k)}\sum_{p=1}^k\abs {\Lambda_p^{m-\ell}(\abs V-k)}\beta_{p\,\ell}(m, k, \abs V),\]
	with $\beta_{p\,\ell}(m, k, \abs V)$, a coefficient.

	The idea behind this formula is explained in introduction of Section~\ref{sec:results}: to find the amount of graphs having $n$ vertices, $m$ edges
	and a largest connected component of size $k$, let's place a connected component of size $k$ somewhere in the graph (so choose $k$ in $\abs V$ vertices), and then multiply this
	by the amount of possible graphs of largest connected component of size $p \in \{1, \ldots, k\}$ (so lower or equal to $k$).
	\end{remark}

	\begin{proof}[Idea of proof to be extended later on] In order to prove the equality of the cardinalities, let's find a bijective function $\Omega$ between
	$\Lambda_k^m(V, \cdot)$ and a set like:
	\[\mathfrak Q_k^m(V) \coloneqq \bigsqcup_{\stackrel {W \in \mathcal P(V)}{\abs W = k}}\bigsqcup_{\ell=k-1}^{\min\left(m, X(k)\right)}
		\Lambda_k^\ell(W, \cdot) \times \left(\bigsqcup_{p=1}^k \Lambda_p^{m-\ell}(V \setminus W, \cdot)\right).\]
	\end{proof}

	\begin{lemma} The sets $\chi_\ell(V, \cdot)$ and $\Lambda_{\abs V}^\ell(V, \cdot)$ are equal.
	\end{lemma}

	\begin{proof} A graph $\Gamma(V, E)$ is connected if and only if its largest connected component contains all its vertices, i.e. $\LCC(\Gamma(V, E)) = V$.

	This is equivalent to say that $\abs {\LCC(\Gamma(V, E))} = \abs V$ since $\forall W \in \mathcal P(V) : \abs W = \abs V \Rightarrow V = W$:
	\[\forall W \in \mathcal P(V) : \abs {\left\{\widetilde W \in \mathcal P(V) \st \abs W = \abs {\widetilde W}\right\}} = \binom {\abs V}{\abs W},\]
	and $\binom {\abs V}{\abs V} = 1$, thus $\left\{W \in \mathcal P(V) \st \abs W = \abs V\right\} = \{V\}$.
	\end{proof}

	\subsection{Decomposing set $\Lambda_k(V)$}
		\begin{definition} For $k \in \N$, and $\alpha \in \N$, we define:
		\[\Lambda_{k,\alpha}(V, \cdot) \coloneqq \left\{\Gamma(V, E) \in \Lambda_k(V, \cdot) \st
			\left\{W \in \mathcal P(V) \st \Delta_W(\Gamma(V, E)) \in \chi(W) \text{ and } \abs W = \abs {\LCC(\Gamma(V, E))}\right\} = \alpha\right\},\]
		the class of all graphs in $\Lambda_k(V, \cdot)$ having exactly $\alpha$ connected components of maximum size.
		\end{definition}

		\begin{remark} Even though several connected components of maximum size do exist in a graph, \textit{the one} $\LCC$ is still defined unambiguously!
		\end{remark}

		\begin{lemma}~
		\begin{enumerate}
			\item For $k > \abs V$ or $k=0$, we have: $\forall \alpha \in \N : \Lambda_{k,\alpha}(V, \cdot) = \emptyset$.
			\item For $k \in \intint 1{\abs V}$ and $\alpha > \floor {\frac {\abs V}k}$, we have $\Lambda_{k,\alpha}(V, \cdot) = \emptyset$.
		\end{enumerate}
		\end{lemma}

		\begin{proof}~
		\begin{enumerate}
			\item For $k > \abs V$ or $k=0$, it is obvious that: $\Lambda_k(V, \cdot) = \emptyset$ (and then $\Lambda_{k,\alpha}(V, \cdot)$).
			\item Take such $k$ and $\alpha$. Assume (\textit{ad absurdum}) that there exists $\Gamma(V, E) \in \Lambda_{k, \alpha}(V, \cdot)$. We have then
			$L_1, \ldots, L_\alpha \in \mathcal P(V)$ such that $\forall i \in \intint 1\alpha : \abs {L_i} = k$. Also, since the $L_i$'s are connected component,
			they are disjoint, i.e. $\forall (i, j) \in \intint 1\alpha^2 : i \neq j \Rightarrow L_i \cap L_j = \emptyset$.

			Thus $\bigcup_{i=1}^\alpha L_i \subset V$, and $\sum_{i=1}^\alpha \abs {L_i} \leq \abs V$. But:
			\[\sum_{i=1}^\alpha\abs {L_i} = \alpha k > \floor {\frac {\abs V}k}k > \frac {\abs V}kk = \abs V,\]
			which leads a contradiction: $\abs V > \abs V$.
			We deduce that $\Lambda_{k,\alpha}(V, \cdot) = \emptyset$.
		\end{enumerate}
		\end{proof}

		\begin{remark} Again, for $m \in \intint 1{X(\abs V)}$, we define $\Lambda_{k,\alpha}^m(V, \cdot)$ by $\Lambda_k^m(V, \cdot) \cap \Lambda_{k,\alpha}(V, \cdot)$.
		\end{remark}

		\begin{corollary}
		\[\forall k < \abs V : \Lambda_k(V, \cdot) = \bigsqcup_{m=k-1}^{X(\abs V)}\bigsqcup_{\alpha=1}^{\floor {\frac {\abs V}k}}\Lambda_{k,\alpha}^m(V, \cdot).\]
		\end{corollary}

		\begin{proof} Unions are trivially disjointed.

		Now show the equality. The right-hand side is trivially included in $\Lambda_k(V, \cdot)$ (by definition of $\Lambda_{k,\alpha}^m(V, \cdot)$).

		Now take $\Gamma(V, E) \in \Lambda_k(V, \cdot)$. We know that $\Lambda_k^{\abs E}(V, \cdot)$ with $\abs E \leq X(\abs V)$. As well, we know that the amount
		of connected components of size $\abs {\LCC(\Gamma(V, E))} = k$ is at least 1 (because $\Gamma(V, E) \in \Lambda_k(V, \cdot)$), and lower or equal to
		$\floor {\frac {\abs V}k}$ by previous Lemma.
		\end{proof}

		From now on, let's write:
		\[\mathfrak Q_{k,1}^m(V) \coloneqq \bigsqcup_{\stackrel {W \in \mathcal P(V)}{\abs W = k}}\bigsqcup_{\ell=k-1}^{\min\left(m, X(k)\right)}
			\chi_\ell(W, \cdot) \times \left(\bigsqcup_{p=1}^{k-1}\Lambda_p^{m-\ell}(V, \cdot)\right).\]

		\begin{proposition}\label{prp:Lambda_k,alpha^m(V, cdot) cong mathfrak Q_k,alpha^m(V)} For $k \in \intint 1{\abs V}$ and $m \in \intint 1{X(\abs V)}$, we have:
		\[\Lambda_{k,1}^m(V, \cdot) \cong \mathfrak Q_{k,1}^m(V),\]
		i.e. there exists a bijection between these two sets.
		\end{proposition}

		\begin{proof} Consider the function defined by:
		\[\Omega : \Lambda_{k,1}^m(V, \cdot) \to \mathfrak Q_{k,1}^m(V) :
			\Gamma(V, E) \mapsto \left(\Delta_{\LCC(\Gamma(V, E))}(\Gamma(V, E)), \Delta_{V \setminus \LCC(\Gamma(V, E))}(\Gamma(V, E))\right).\]

		We know that $\Omega(\Gamma(V, E))$ gives a partition of $\Gamma(V, E)$ because there doesn't exist $e = \{v_{i_1}, v_{i_2}\} \in E$ such that
		$v_{i_1} \in \LCC(\Gamma(V, E))$ and $v_{i_2} \in V \setminus \LCC(\Gamma(V, E))$ by definition of connected component. Thus if:
		\[\left(\Gamma_1(\LCC(\Gamma(V, E)), E_{\LCC}), \Gamma_2(V \setminus \LCC(\Gamma(V, E)), E_{V \setminus \LCC})\right) = \Omega(\Gamma(V, E)),\]
		we have $E_{\LCC} \sqcup E_{V \setminus \LCC} = E$.

		Let's prove that $\Omega$ is injective. Take $\Gamma_1(V, E_1)$ and $\Gamma_2(V, E_2)$ in $\Lambda_k^m(V, \cdot)$. Assume that
		$\Omega(\Gamma_1(V, E_1)) = \Omega(\Gamma_2(V, E_2))$. This implies that $\LCC(\Gamma_1(V, E_1)) = \LCC(\Gamma_2(V, E_2))$ (and thus that their complementary
		in $V$ are equal as well) and that the restriction of $\Gamma_1(V, E_1)$ and $\Gamma_2(V, E_2)$ to their common LCC are equal. Their restriction to the
		complementary of the common LCC in $V$ are equal as well.

		Thus $\Gamma_1(V, E_1)$ and $\Gamma_2(V, E_2)$ have the same partition and are then equal (i.e. $E_1=E_2$).

		Now, let's prove that $\Omega$ is surjective. For $W \in \mathcal P(V)$, take:
		\[\left(\Gamma_1(W, E_1), \Gamma_2(V \setminus W, E_2)\right) \in \mathfrak Q_{k,1}^m(V).\]

		We deduce that if $\ell \coloneqq \abs {E_1}$, we have $\abs {E_2} = m-\ell$. Also, $\abs {\LCC(\Gamma_2(V \setminus W, E_2))} \lneqq k$ and
		$\Gamma_1(W, E_1) \in \chi_\ell(W)$.

		For $\Gamma(V, E) \coloneqq \Gamma(V \setminus W \sqcup W, E_1 \sqcup E_2)$, we have indeed that $\Gamma(V, E) \in \Lambda_k^m(V, \cdot)$ and
		$\Omega(\Gamma(V, E)) = \left(\Gamma_1, \Gamma_2\right)$ because $\abs {\LCC(\Gamma(V, E))} = \abs {W} = k$ since $W$ is the only connected
		component of size $k$.
		\end{proof}

		\begin{definition} For $k, \alpha \in \N^*$, let's define:
		\[\mathcal P_{k,\alpha}(V) \coloneqq \left\{\{W_1, \ldots, W_\alpha\} \in \mathcal P\left(\mathcal P(V)\right) \st
			\begin{cases}
					&\forall i \in \intint 1\alpha : \abs {W_i} = k \\
					&\forall (i, j) \in \intint 1\alpha^2 : i \neq j \Leftrightarrow W_i \cap W_j = \emptyset
				\end{cases}
		\right\},\]
		thus $\mathcal P_{k,\alpha}(V)$ is the set of all sets containing $\alpha$ subsets of $V$ which are disjointed and of size $k$.
		\end{definition}

		\begin{remark} We can tell:
		\[\abs {\mathcal P_{k,\alpha}(V)} = \frac 1{\alpha!}\prod_{\beta=0}^{\alpha-1}\binom {\abs V-k\beta}k = \frac 1{\alpha!}\frac {\abs V!}{(k!)^\alpha(\abs V-k\alpha)!}.\]
		\end{remark}

		\begin{definition} For $k \in \intint 1{\abs V}, m \in \intint 0{X(\abs V)}$, and $\alpha \in \intint 2{\floor {\frac {\abs V}k}}$, let's define:
		\begin{align*}
			\mathfrak Q_{k,\alpha}^m(V) &\coloneqq
					\bigsqcup_{\stackrel {\{W_1, \ldots, W_\alpha\} \in \mathcal P_{k,\alpha}(V)}{\mu(W_1) < \ldots < \mu(W_\alpha)}}
						\bigsqcup_{\stackrel {(i_1, \ldots, i_\alpha) \in \intint {k-1}{X(k)}^\alpha}{\st \sum_{j=1}^\alpha i_j \leq \min\left(m, \alpha X(k)\right)}}
						\left[\left(\prod_{j=1}^\alpha \chi_{i_j}(W_j, \cdot)\right) \times \left(\bigsqcup_{p=1}^{k-1}
							\Lambda_p^{m-\sum_{j=1}^\alpha i_j}\left(V \setminus \bigsqcup_{j=1}^\alpha W_j, \cdot\right)\right)\right] \\
				&= \bigsqcup_{\stackrel {\{W_1, \ldots, W_\alpha\} \in \mathcal P_{k,\alpha}(V)}{\mu(W_1) < \ldots < \mu(W_\alpha)}}
					\bigsqcup_{\Sigma=\alpha(k-1)}^{\min\left(m, \alpha X(k)\right)}
					\bigsqcup_{\stackrel {(i_1, \ldots, i_\alpha) \in \intint {k-1}{X(k)}^\alpha}{\st \sum_{j=1}^\alpha i_j = \Sigma}}
						\left[\left(\prod_{j=1}^\alpha \chi_{i_j}(W_j, \cdot)\right) \times \left(\bigsqcup_{p=1}^{k-1}
							\Lambda_p^{m-\Sigma}\left(V \setminus \bigsqcup_{j=1}^\alpha W_j, \cdot\right)\right)\right]
		\end{align*}
		\end{definition}

		\begin{theorem} For $(k, m) \in \intint 1{\abs V} \times \intint 0{X(\abs V)}$ and $\alpha \in \intint 1{\floor {\frac {\abs V}k}}$, we have:
		\[\Lambda_{k,\alpha}^m(V, \cdot) \cong \mathfrak Q_{k,\alpha}^m(V).\]
		\end{theorem}

		\begin{proof} For $\alpha=1$, the theorem is proven by Proposition~\ref{prp:Lambda_k,alpha^m(V, cdot) cong mathfrak Q_k,alpha^m(V)}.\footnote{This Proposition is not exactly
		expressed the same way because $\mathfrak Q_{k,1}^m(V)$ does not consider a set $\{W_1\} \in \mathcal P\left(\mathcal P(V)\right)$ but only a subset $W \in \mathcal P(V)$.
		Yet, the proof is equivalent.}

		Now, for $\alpha \geq 2$, we have the function:
		\begin{align*}
			\Omega_\alpha &: \Lambda_{k,\alpha}^m(V, \cdot) \to \mathfrak Q_{k,\alpha}^m(V) : \\
			&\Gamma(V, E) \mapsto \left(\Delta_{W_1}(\Gamma(V, E)), \ldots, \Delta_{W_\alpha}(\Gamma(V, E)), \Delta_{V \setminus \bigsqcup_{j=1}^\alpha W_j}(\Gamma(V, E))\right),
		\end{align*}
		for $W_1, \ldots, W_\alpha$ the subsets of $V$ two by two disjoints, such that $\forall i \in \intint 1\alpha : \abs {W_i}=k$, and that:
		\[\forall i \in \intint 2\alpha : \mu(W_{i-1}) \lneqq \mu(W_i).\]

		We know that $W_1, \ldots, W_\alpha$ are the only connected components of size $k$ because $\Gamma(V, E) \in \Lambda_{k,\alpha}^m(V, \cdot)$. And also,
		values of $\mu(W_j)$ can't be equal for different indices by definition of connected components. This implies that function $\Omega_\alpha$ is properly
		defined.

		Now, prove that $\Omega_\alpha$ is bijective.

		\paragraph{Injective} Take $\Gamma_1(V, E_1), \Gamma_2(V, E_2) \in \Lambda_{k,\alpha}^m(V, \cdot)$. Let's assume that:
		\[\Omega_\alpha(\Gamma_1(V, E)) = \Omega_\alpha(\Gamma_2(V, E_2)).\]

		We can deduce that $\Gamma_1$ and $\Gamma_2$ have the same connected components, and that their restrictions to these connected components are equal as well.
		By similar argument than in proof of Proposition~\ref{prp:Lambda_k,alpha^m(V, cdot) cong mathfrak Q_k,alpha^m(V)}, we find that $\Gamma_1$ and $\Gamma_2$ must be equal.

		\paragraph{Surjective} Take:
		\[\left(\Gamma_1(W_1, E_1), \ldots, \Gamma_\alpha(W_\alpha, E_\alpha), \Gamma(W, E)\right) \in \mathfrak Q_{k,\alpha}^m(V),\]
		and proof that there exists a graph $\hat \Gamma(V, \hat E) \in \Lambda_{k,\alpha}^m(V, \cdot)$ such that:
		\[\Omega_\alpha(\hat \Gamma(V, \hat E)) = \left(\Gamma_1(W_1, E_1), \ldots, \Gamma_\alpha(W_\alpha, E_\alpha), \Gamma(V, E)\right).\]

		We know that $\{W_1, \ldots, W_\alpha\} \in \mathcal P_{k,\alpha}(V)$ by definition of $\mathfrak Q_{k,\alpha}^m(V)$, and that:
		\[V = W \sqcup \bigsqcup_{j=1}^\alpha W_j.\]

		As well, by definition of $\mathfrak Q_{k,\alpha}^m(V)$, we know that if $\hat E \coloneqq E \sqcup \bigsqcup_{j=1}^\alpha E_j$, then $\abs {\hat E} = m$.
		So $\hat \Gamma(V, \hat E)$, the graph created by \textit{assembling} the different components $\Gamma_1$ to $\Gamma_\alpha$ and $\Gamma$, is indeed in
		$\Lambda_{k,\alpha}^m$ because:
		\[\forall j \in \intint 1\alpha : \abs {\LCC(\Gamma_j(W_j, E_j)} = k\qquad \text{ and } \abs {\LCC(\Gamma(V, E))} \lneqq k\]
		by definition of $\mathfrak Q_{k,\alpha}^m(V)$.

		Finally, $\Omega_\alpha(\hat \Gamma(V, \hat E))$ yields indeed $(\Gamma_1, \ldots, \Gamma_\alpha, \Gamma)$ since connected components are ordered according
		to the $\mu$ function defined in Definition~\ref{def:mu function}
		\end{proof}

		\begin{remark} The problem of the largest connected component size has been reduced to a problem of connected graphs and recursive combinatorics.

		Recursive values like these ones can be computed pretty efficiently thanks to dynamic programming.
		\end{remark}

\section{Conclusion}
	We have then proved that the cardinality of set $\Lambda_k^m$ is equal to:
	\begin{align*}
		\abs {\Lambda_k^m(V, \cdot)} &= \abs {\bigsqcup_{\alpha=1}^{\floor {\frac {\abs V}k}}\Lambda_{k,\alpha}^m(V, \cdot)}
			= \sum_{\alpha=1}^{\floor {\frac {\abs V}k}}\abs {\Lambda_{k,\alpha}^m(V, \cdot)} = \sum_{\alpha=1}^{\floor {\frac {\abs V}k}}\abs {\mathfrak Q_{k,\alpha}^m(V)} \\
		&= \sum_{\alpha=1}^{\floor {\frac {\abs V}k}}\abs {\mathcal P_{k,\alpha}(V)}\sum_{\Sigma=\alpha(k-1)}^{\min(m, \alpha X(k))}
			\sum_{\stackrel {(i_1, \ldots, i_\alpha) \in \intint {k-1}{X(k)}^\alpha}{\sum_{j=1}^\alpha i_j = \Sigma}}
			\left(\prod_{j=1}^\alpha\abs {\chi_{i_j}(k)} \times \sum_{p=1}^{k-1}\abs {\Lambda_p^{m-\Sigma}(\abs V-k\alpha)}\right),
	\end{align*}
	from which we eventually deduce:
	\[\forall k \in \intint 1{\abs V} : \P\left[\abs {\LCC(\mathscr G(V, m))} = k\right] = \frac {\abs {\Lambda_k^m(V, \cdot)}}{\abs {\Gamma(V, \cdot)}}.\]

\end{document}
