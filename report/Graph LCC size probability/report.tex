\documentclass{article}

\usepackage[english]{babel}
\usepackage[utf8]{inputenc}
\usepackage[T1]{fontenc}
\usepackage{fullpage}
\usepackage[parfill]{parskip}

% math
\usepackage{amsmath}
\usepackage{amssymb}
\usepackage{amsfonts}
\usepackage{amsthm}
\usepackage{commath}
\usepackage{stmaryrd}
\usepackage{mathtools}
\usepackage{mathrsfs}

\newtheorem{lemma}{Lemma}[section]
\newtheorem{corollary}[lemma]{Corollary}
\newtheorem{theorem}[lemma]{Theorem}
\theoremstyle{definition}
\newtheorem{definition}[lemma]{Definition}
\theoremstyle{remark}
\newtheorem*{remark}{Remark}
 % operators
\DeclareMathOperator{\LCC}{LCC}
\DeclareMathOperator*{\argmax}{arg\,max}

 % commands
\renewcommand{\P}{\mathbb P}

\newcommand{\R}{\mathbb R}
\newcommand{\N}{\mathbb N}
\newcommand{\intint}[2]{\left\llbracket#1, #2\right\rrbracket}
\newcommand{\st}{\text{ s.t.}}

\author{Robin Petit}
\date{March-April 2017}
\title{On the distribution of the largest connected component size in random graphs with fixed edges set size}

\begin{document}
\maketitle
\tableofcontents

\section{Introduction}
	\subsection{Definitions and preliminary results}
		Let's consider $V = \{v_1, \ldots, v_{\abs V}\}$ a set of vertices. We denote by $\abs V$ the cardinality of the set $V$. Let's define
		the function:
		\[X : \N \to \N : n \mapsto \frac {n(n-1)}2.\]

		\begin{definition} An undirected graph $\Gamma$ is denoted $\Gamma(V, E)$ for $V$ its vertices set, and $E$ its edges set, with $E = \{e_1, \ldots, e_{\abs E}\}$
		and $\forall i \in \intint 1{\abs E} : e_i = \{v_{i1}, v_{i2}\}$ for $1 \leq i_1, i_2 \leq \abs V$.
		\end{definition}

		\begin{remark} $\abs E$ is usually denoted as $m$, and $\abs V$ is sometimes denoted as $n$. Both these numbers are (non-strictly) positive integers.
		\end{remark}

		\begin{definition} The set of all the existing graphs having given vertices set $V$ is denoted by $\Gamma(V, \cdot)$. We denote $\Gamma_m(V, \cdot)$
		the subset of $\Gamma(V, \cdot)$ such that $\abs E = m$.
		\end{definition}

		\begin{remark} We observe that:
		\[\Gamma(V, \cdot) = \bigsqcup_{m \in \N}\Gamma_m(V, \cdot).\]
		\end{remark}

		\begin{definition} For every $n \in \N$, we define $\mathcal K_n$ as the \textit{complete graph} of size $n$.
		\end{definition}

		\begin{lemma} For a graph $\Gamma(V, E)$, we have $\abs E \leq X(\abs V)$.
		\end{lemma}

		\begin{proof} We know that $\Gamma(V, E) \leq \mathcal K_{\abs V}$, and $\mathcal K_{\abs V}$ has exactly $X(\abs V)$ edges (vertex $v_i$ is connected
		to vertices $v_{i+1}$ to $v_{\abs V}$, so the number of edges is equal to $\sum_{i=1}^{\abs V}(\abs V - i) = \sum_{i=0}^{\abs V - 1}i = X(\abs V)$).
		\end{proof}

		\begin{lemma} For given vertices set $V$ and fixed number of edges $m \in \N$, we have:
		\[\abs {\Gamma_m(V, \cdot)} = \begin{cases}\binom {X(\abs V)}m &\text{ if } m \leq X(\abs V) \\0 &\text{ else}\end{cases}.\]
		\end{lemma}

		\begin{corollary} For given vertices set $V$, we have $\abs {\Gamma(V, \cdot)} = 2^{X(\abs V)}$.
		\end{corollary}

		\begin{proof} Since $\Gamma(V, \cdot)$ is given by a disjoint union over $m$, its cardinality is equal to the sum of the individual cardinalities:
		\[\abs {\Gamma(V, \cdot)} = \sum_{m \in \N}\abs {\Gamma_m(V, \cdot)} = \sum_{k=0}^{X(\abs V)}\abs {\Gamma_m(V, \cdot)} = \sum_{k=0}^{X(\abs V)}\binom {X(\abs V)}m = 2^{X(\abs V)}.\]
		\end{proof}

		\begin{definition} A graph $\Gamma(V, E)$ is said to be connected if for each $v, w \in V$, there exists a path between $v$ and $w$. We denote by $\chi(V, \cdot)$ the
		set of all connected graphs having vertices set $V$. Again, for $m \in \N$, we denote by $\chi_m(V, \cdot) \subset \chi(V, \cdot)$ the set of connected graphs having $m$ edges.
		\end{definition}

		\begin{remark} $\chi(V, \cdot) \subset \Gamma(V, \cdot)$, and:
		\[\chi(V, \cdot) = \bigsqcup_{m \in \N}\chi_m(V, \cdot).\]
		\end{remark}

		\begin{lemma} For $m < \abs V$ or $m > X(\abs V)$, we have $\abs {\chi_m(V, \cdot)} = 0$.
		\end{lemma}

		\begin{definition} For every $W \in \mathcal P(V)$, we define $\Delta_W : \Gamma(V, \cdot) \to \Gamma(W, \cdot) : \Gamma(V, E) \mapsto \Gamma'(W, E')$
		such that:
		\[E' = \left\{\{v_i, v_j\} \in E \st v_i, v_j \in W\right\}.\]
		\end{definition}

		\begin{definition} We define the \textit{connected component of vertex $v_i \in V$ in graph $\Gamma(V, E)$} by the biggest subset (in the sense of inclusion) $W$
		of $V$ such that $\Delta_W(\Gamma(V, E)) \in \chi(W, \cdot)$.

		We then define the \textit{largest connected component of the graph $\Gamma(V, E)$} as:
		\[\LCC(\Gamma(V, E)) \coloneqq \argmax_{\stackrel {W \in \mathcal P(W)}{\Delta_W(\Gamma(V, E) \in \chi(W, \cdot)}}\abs W
			= \argmax_{W \in \mathcal P(V)}\abs W\mathbb I_{\left[\Delta_W(\Gamma(V, E) \in \chi(W, \cdot)\right]}.\]

		The set $\Lambda_k^m(V, \cdot)$ is then the set of all graphs $\Gamma(V, E) \in \Gamma(V, \cdot)$, such that $\abs E = m$ and $\abs {\LCC(\Gamma(V, E))} = k$.
		\end{definition}

		\begin{remark} Since $\Lambda_k(V, \cdot) = \bigsqcup_{m=0}^{X(\abs V)}\Lambda_k^m(V, \cdot)$ and:
		\[\Gamma(V, \cdot) = \bigsqcup_{k=1}^{\abs V}\bigsqcup_{m=0}^{X(\abs V)}\Lambda_k^m(V, \cdot),\]
		we want to know what is $\abs {\Lambda_k^m(V, \cdot)}$ equal to.
		\end{remark}

		\begin{definition} Let's declare a new random variable $\mathscr G(V)$, a graph uniformly distributed in $\Gamma(V, \cdot)$, thus such that:
		\[\forall \Gamma(V, E) \in \Gamma(V, \cdot) : \P[\mathscr G(V) = \Gamma(V, E)] = \frac 1{\abs {\Gamma(V, \cdot)}} = 2^{-X(\abs V)}.\]
		\end{definition}

	\subsection{Objectives}
		The objective now is to find an expression for $\abs {\Lambda_k(V, \cdot)}$ since we are looking for:
		\[\P[\LCC(\mathscr G(V)) = k] = \frac {\abs {\Lambda_k(V, \cdot)}}{\abs {\Gamma(V, \cdot)}} = \frac 1{\abs {\Gamma(V, E)}}\sum_{m=0}^{X(\abs V)}\abs {\Lambda_k^m(V, \cdot)}.\]

		Let's denote this value $p_k \coloneqq \P\left[\abs {\LCC(\mathscr G(V)} = k\right].$

\section{Results}
\end{document}
